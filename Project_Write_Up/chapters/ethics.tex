\chapter{Project Ethics}
\label{ch:ethics}

Ethical considerations are essential in any scientific investigation, particularly when using real-world data or building systems that may influence financial decision-making. This project fully adheres to the University of Liverpool’s ethical guidelines for student research and satisfies all requirements for responsible data use, privacy, model transparency and accurate reporting.

\section{Data Ethical Compliance}
This project uses only public, non-personal market data downloaded from Yahoo Finance. The dataset consists of:
\begin{itemize}
    \item historical OHLCV price series,
    \item publicly accessible stock-market information,
    \item non-sensitive numerical variables derived through feature engineering.
\end{itemize}

No private, personal or commercially restricted datasets were accessed at any stage. According to the University of Liverpool’s ethical classification system, this work falls under:
\begin{itemize}
    \item \textbf{Category A0}: No human participants,
    \item \textbf{Category C2}: Use of publicly available, non-sensitive secondary data.
\end{itemize}

Research in these categories does not require formal ethical approval, and all activities remain within standard academic guidance.

\section{Privacy and Anonymity}
The dataset contains:
\begin{itemize}
    \item no personal identifiers,
    \item no user-specific information,
    \item no sensitive demographic or behavioural attributes.
\end{itemize}

All modelling inputs and outputs relate solely to aggregated, publicly available financial instruments; therefore:
\begin{itemize}
    \item no personal data were collected, stored or processed,
    \item there is no risk of identifying individuals,
    \item no data protection or GDPR concerns apply.
\end{itemize}

Furthermore, in accordance with University of Liverpool guidelines, all personal identifiers (author name, student ID, acknowledgements) appear only in the initial, non-anonymous front matter of the dissertation and are excluded from the anonymous body of the document.

\section{Responsible Use of AI Models}
The models developed in this dissertation analyse historical financial data for academic purposes only. No model output should be interpreted as:
\begin{itemize}
    \item investment advice,
    \item trading recommendations,
    \item actionable or deployable financial signals.
\end{itemize}

Care has been taken to:
\begin{itemize}
    \item avoid overstating model performance,
    \item discuss limitations of machine learning in financial forecasting,
    \item highlight risks such as overfitting, non-stationarity and structural instability,
    \item emphasise the dangers of algorithmic misuse.
\end{itemize}

All experiments were conducted in a controlled research environment and are not intended for live trading systems without substantial further validation.

\section{Transparency and Reproducibility}
The project adheres to best practices for reproducible research. This includes:
\begin{itemize}
    \item full documentation of data sources,
    \item modular, openly accessible Python code,
    \item transparent preprocessing and feature-engineering steps,
    \item clearly reported evaluation metrics and modelling assumptions,
    \item plots, tables and figures generated directly from reproducible notebook outputs.
\end{itemize}

No proprietary algorithms or undisclosed methods were used.

\section{Model Bias, Fairness and Financial Stability}
Although fairness issues typically relate to models affecting individuals (e.g.\ credit scoring), financial forecasting raises different ethical risks, including:
\begin{itemize}
    \item market manipulation through automated strategies,
    \item volatility amplification via algorithmic feedback loops,
    \item exploitation of high-frequency microstructure patterns.
\end{itemize}

This project avoids such risks by:
\begin{itemize}
    \item using only daily-frequency data,
    \item avoiding live, high-frequency or automated trading strategies,
    \item focusing solely on educational, analytical and comparative evaluation,
    \item ensuring that no models operate in real-world financial environments.
\end{itemize}

The work evaluates model behaviour for academic insight only, with no real-world financial impact.

\section{Statement of Ethical Compliance}
This project complies fully with the University of Liverpool’s ethical guidance for student research. It involved exclusively public, non-sensitive financial data (A0/C2 classification), required no interaction with human participants, and introduced no foreseeable ethical, legal or privacy risks. All analyses were conducted responsibly, transparently and strictly for academic purposes.
