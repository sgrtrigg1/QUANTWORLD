\chapter*{Introduction}
\addcontentsline{toc}{chapter}{Introduction}
\label{ch:introduction}

Artificial Intelligence (AI) and Machine Learning (ML) have become essential to modern quantitative finance. Financial markets produce vast amounts of structured and unstructured data, including prices, trading volumes, corporate disclosures, macroeconomic indicators, limit-order book updates, and high-frequency tick data. Extracting meaningful insights from such data requires computational tools capable of modelling non-linearity, managing noise, adapting to changing market regimes, and learning temporally dependent patterns. Traditional econometric models, while highly interpretable, often struggle to capture the complexity and dynamics of modern markets. Consequently, financial institutions increasingly depend on AI-driven methods that deliver improved predictive power, better risk modelling, and more flexible decision-support systems. This dissertation explores the application of artificial intelligence techniques—including supervised learning, unsupervised learning, deep learning, and reinforcement learning—to the task of forecasting and analysing financial time series. Special emphasis is placed on short-term price prediction and market-regime identification, as these tasks form the foundation of many quantitative trading and risk-management strategies. By assessing a range of models from classical linear regression to modern deep neural networks, this work investigates how model complexity, data representation, and temporal structure influence predictive accuracy and robustness.

A central challenge addressed in this dissertation is the highly noisy, non-stationary nature of financial time series. Market dynamics evolve across time, exhibiting features such as structural breaks, momentum–reversal cycles, volatility clustering and regime shifts. High-capacity AI models may overfit to these fluctuations, learning patterns that fail to generalise out-of-sample, whereas under-powered models risk oversimplifying underlying structure. This trade-off is explored through the bias–variance decomposition, using sliding-window Ordinary Least Squares (OLS) regression as a case study. The resulting empirical investigation demonstrates how window length, model flexibility and feature design shape forecasting behaviour in volatile markets. The primary aim of this dissertation is to evaluate how different machine-learning and deep-learning approaches perform when applied to real financial data under a unified and reproducible pipeline. This includes:
\begin{itemize}
    \item assessing the stability and robustness of linear, nonlinear and neural models under varying market conditions;
    \item analysing supervised classification methods for market-regime detection;
    \item applying unsupervised techniques such as K-Means clustering and Principal Component Analysis (PCA) to uncover latent structure within market data;
    \item building deep learning models (ANNs and LSTMs) for sequential forecasting;
    \item exploring reinforcement learning as an optional extension for sequential trading decisions.
\end{itemize}

The contribution of this dissertation is threefold. First, it provides a fully reproducible modelling pipeline implemented in Python, covering data collection, cleaning, feature engineering, rolling-window construction, model training and performance evaluation. Second, it offers a comparative analysis of classical and modern AI methods on the same dataset, allowing direct examination of strengths and weaknesses across techniques. Third, it provides an academically grounded discussion of why certain models perform well—or poorly—in financial contexts, drawing links to concepts such as non-stationarity, noise dominance, model flexibility and temporal dependencies.

The remainder of this dissertation is organised as follows. Section~\ref{sec:case-studies} introduces a set of case-study scenarios that illustrate real-world applications of AI across financial services, including NLP-driven sentiment analysis, anomaly detection, risk forecasting and recommendation systems. Section~\ref{sec:methodology} outlines the methodology and implementation pipeline, describing the data sources, preprocessing steps, feature engineering, exploratory analysis and model architecture. Section~\ref{sec:supervised-learning} evaluates supervised learning approaches for regression and classification, including an empirical exploration of the bias–variance trade-off. Section~\ref{sec:unsupervised-learning} covers unsupervised learning techniques for regime discovery, anomaly detection and dimensionality reduction. Section~\ref{sec:deep-learning} introduces deep learning architectures and their application to sequential financial data, with a focus on ANNs and LSTMs. Section~\ref{sec:reinforcement-learning} presents an optional reinforcement learning framework for trading and portfolio decisions. Section~\ref{sec:results-discussion} provides evaluation, testing and discussion of results. Sections~\ref{sec:ethics} and~\ref{sec:conclusion} address project ethics, conclusions and avenues for future work, followed by BCS criteria, references and appendices.

This introduction establishes the context, motivation and academic rationale for applying AI techniques in quantitative finance. The following sections develop these themes further through theory, implementation and empirical analysis.
