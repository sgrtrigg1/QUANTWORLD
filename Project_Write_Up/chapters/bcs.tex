\chapter{BCS Criteria and Self-Reflection}
\label{ch:bcs}

This chapter demonstrates how the project meets the competencies expected by the British Computer Society (BCS) for Master’s-level work. It also contains a critical self-reflection on the development process, challenges encountered and skills gained throughout the dissertation.

\section{BCS Learning Outcomes}

\subsection{Application of Practical and Analytical Skills}
This project applied a wide range of practical and analytical skills developed throughout the MSc programme, including:
\begin{itemize}
    \item data collection and cleaning using Python,
    \item feature engineering and exploratory data analysis,
    \item implementation of supervised, unsupervised and deep-learning models,
    \item application of appropriate statistical and machine-learning evaluation metrics,
    \item modular software development using class-based Python design,
    \item critical interpretation of model behaviour in non-stationary environments.
\end{itemize}

By developing a complete modelling pipeline—from raw data ingestion to model evaluation—the work demonstrates strong technical proficiency and analytical capability.

\subsection{Innovation and Creativity}
The project exhibits creativity through:
\begin{itemize}
    \item integrating classical econometric techniques with modern AI models,
    \item applying sliding-window OLS to explore bias–variance dynamics,
    \item combining supervised and unsupervised learning for regime discovery,
    \item evaluating deep-learning architectures within a domain constrained by noise,
    \item including a reinforcement-learning framework to illustrate sequential decision-making.
\end{itemize}

The design of the modelling pipeline reflects innovative thinking, enabling modularity, reproducibility and systematic experimentation.

\subsection{Synthesis of Information and Quality of Solution}
The dissertation synthesises concepts from finance, machine learning, deep learning, statistics and time-series analysis. It delivers:
\begin{itemize}
    \item a unified modelling framework,
    \item consistent evaluation across model families,
    \item a comparative analysis of strengths and weaknesses,
    \item insights into the limits of financial predictability,
    \item a theoretically grounded interpretation of empirical results.
\end{itemize}

The quality of the solution lies not only in successful implementation, but also in the critical understanding of when and why specific models perform well—or poorly—in financial contexts.

\subsection{Meeting a Real Need in a Wider Context}
AI-driven forecasting and regime analysis are increasingly relevant to financial institutions, investment firms and regulatory bodies. The project addresses real-world needs such as:
\begin{itemize}
    \item recognising market regimes for risk management,
    \item understanding model limitations under noise and non-stationarity,
    \item providing a scalable, modular pipeline for exploratory modelling,
    \item evaluating the suitability of deep learning for practical financial forecasting.
\end{itemize}

Although academic in nature, the methodology and findings closely mirror challenges encountered in industry practice.

\subsection{Ability to Self-Manage a Significant Piece of Work}
The dissertation required substantial self-management, including:
\begin{itemize}
    \item independent planning of the project structure,
    \item organising data collection, cleaning and code development,
    \item scheduling iterative modelling and evaluation stages,
    \item managing time alongside other academic and personal commitments,
    \item maintaining version control and structured directories for reproducibility.
\end{itemize}

Managing diverse components—data engineering, model implementation, analysis and interpretation—demonstrates maturity in self-directed research and time management.

\subsection{Critical Self-Evaluation of the Process}
The following section provides a critical reflection on the strengths, challenges and areas for improvement identified during the project.

\section{Critical Self-Reflection}

\subsection*{Strengths}
\begin{enumerate}
    \item \textbf{Strong conceptual integration:} Successfully merged classical econometrics with modern AI techniques, producing a coherent dissertation bridging theory and practice.
    \item \textbf{Robust software engineering:} The modular pipeline, class-based structure and directory organisation improved clarity, scalability and reproducibility.
    \item \textbf{High-quality analysis and visualisation:} Rolling-window diagnostics, PCA projections, regime maps and clustering plots provided deep empirical insight.
    \item \textbf{Evidence-based reasoning:} Models were compared fairly and conclusions were grounded in observed behaviour rather than optimistic assumptions.
    \item \textbf{Technical growth:} Substantial improvement in Python, modelling workflows, deep-learning implementation and automated analysis.
\end{enumerate}

\subsection*{Challenges and Weaknesses}
\begin{enumerate}
    \item \textbf{Model complexity vs.\ data limitations:} Initial expectations for deep learning were overly optimistic; experimentation revealed that noisy daily data restrict complex models more than anticipated.
    \item \textbf{Time spent on non-essential experiments:} Early exploration of overly complex models (e.g., high-degree polynomials, deep MLPs) provided limited value and highlighted the need for domain-aligned modelling choices.
    \item \textbf{Interpreting poor model results:} Understanding and explaining weak LSTM or ANN performance was initially challenging, but ultimately strengthened the analytical depth of the dissertation.
    \item \textbf{Balancing breadth and depth:} Covering supervised, unsupervised, deep-learning and RL models required careful planning to avoid fragmentation and maintain narrative coherence.
\end{enumerate}

\subsection*{What I Would Do Differently}
\begin{enumerate}
    \item Use higher-frequency data to unlock deeper temporal patterns for LSTMs and RL models.
    \item Employ walk-forward or online learning rather than static train/test splits.
    \item Begin evaluation earlier to accelerate iteration and refine model selection.
    \item Apply systematic hyperparameter optimisation (e.g., Bayesian tuning) for ANN and LSTM stability.
\end{enumerate}

\section{Personal and Professional Development}
This project strengthened skills in:
\begin{itemize}
    \item quantitative modelling and statistical analysis,
    \item Python programming and modular software design,
    \item implementing and evaluating deep learning architectures,
    \item critical thinking and academic communication,
    \item independent research, planning and structured problem-solving.
\end{itemize}

These competencies align directly with roles in data science, quantitative finance, risk analytics and applied machine learning, and provide a strong foundation for future professional development.
