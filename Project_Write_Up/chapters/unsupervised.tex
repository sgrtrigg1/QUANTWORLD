\chapter{Unsupervised Learning}
\label{ch:unsupervised}

Unsupervised learning techniques uncover structure within data without requiring predefined labels. This makes them especially valuable in financial markets, where regime boundaries are ambiguous, anomalies are rare and predictive targets are inherently noisy. Unlike supervised models—which learn mappings from features to known outputs—unsupervised methods seek patterns, clusters or latent factors directly from the data. This chapter introduces the unsupervised models used in this dissertation and demonstrates their value for regime identification, dimensionality reduction and anomaly detection.

\section{Clustering of Market Regimes with K-Means}
Financial markets frequently alternate between phases of trending, mean-reverting, volatile and stable behaviour. These structural shifts—commonly referred to as market regimes—are not explicitly labelled in historical datasets. K-Means clustering provides a simple and interpretable approach to discovering such hidden states.

\subsection{Feature Space for Regime Discovery}
To identify regimes, clusters were learned from the following engineered features:
\begin{itemize}
    \item daily returns,
    \item rolling volatility,
    \item moving-average deviation (Close -- SMA),
    \item RSI (momentum oscillator),
    \item normalised trading volume,
    \item rate of change (ROC).
\end{itemize}

These variables capture price dynamics, risk conditions and short-term momentum. Combined, they form an informative representation of market microstructure without imposing any assumptions about labels.

\subsection{K-Means Algorithm Application}
The K-Means algorithm partitions observations into $k$ clusters by minimising within-cluster variance. For financial regime modelling, $k = 3$ was selected to reflect three economically interpretable states:
\begin{itemize}
    \item bullish / upward-trending regimes,
    \item bearish / downward-trending regimes,
    \item sideways / neutral or choppy regimes.
\end{itemize}

Although $k$ can be chosen using elbow or silhouette scores, a three-cluster structure aligns well with common financial patterns and facilitates comparison with supervised classifiers introduced earlier.

\subsection{Interpretation of Clusters}
The resulting clusters exhibited meaningful separation:
\begin{itemize}
    \item one cluster demonstrated low volatility and positive returns, corresponding to bull regimes;
    \item another exhibited elevated volatility and predominantly negative returns, consistent with bear markets;
    \item a third cluster displayed neutral returns with low-to-medium volatility, indicative of sideways or consolidation periods.
\end{itemize}

These unsupervised clusters corresponded closely to the logistic-regression regime labels defined earlier in Chapter~\ref{ch:supervised}. The alignment between supervised and unsupervised classifications provides strong evidence that regime structure is embedded in the underlying data.

\section{Principal Component Analysis (PCA) for Dimensionality Reduction}
Financial datasets often contain highly correlated features, which can reduce model stability and inflate computational cost. Principal Component Analysis (PCA) provides a principled method for reducing dimensionality while retaining the most informative sources of variance.

\subsection{PCA on Technical Indicators and Returns}
PCA was applied to a matrix of engineered variables including:
\begin{itemize}
    \item returns,
    \item volatility measures,
    \item moving-average distances,
    \item oscillators (RSI, Stochastic),
    \item momentum indicators (ROC),
    \item volume-based metrics.
\end{itemize}

The first few principal components captured a substantial proportion of total variance:
\begin{itemize}
    \item \textbf{PC1 (trend factor)}: dominated by moving averages, returns and ROC;
    \item \textbf{PC2 (volatility factor)}: dominated by rolling standard deviation and ATR;
    \item \textbf{PC3 (liquidity/momentum factor)}: influenced by volume and oscillators.
\end{itemize}

These loadings show that PCA extracts economically interpretable latent drivers.

\subsection{Visualising Market Structure}
When projected onto PC1 and PC2, market observations formed visually separable clusters corresponding to bull, bear and sideways regimes (Figure~\ref{fig:pca_clusters}). These groupings closely mirrored the K-Means assignments, suggesting that regime structure is strongly embedded in the top principal components.

This finding supports two important conclusions:
\begin{enumerate}
    \item market structure is lower-dimensional than its raw feature space, and
    \item regime states are closely tied to underlying variance factors.
\end{enumerate}

\section{Volatility--Return Regime Discovery}
The volatility--return plane is one of the most informative representations in finance. Regimes can be identified by locating observations within combinations such as:
\begin{itemize}
    \item low volatility and positive returns,
    \item high volatility and negative returns,
    \item medium volatility and choppy returns.
\end{itemize}

Using K-Means on the two-dimensional (return, volatility) space:
\begin{itemize}
    \item a stable low-volatility, positive-return cluster emerged—typical of calm bull phases;
    \item a high-volatility, negative-return cluster appeared—indicative of crisis-like behaviour;
    \item a medium-volatility, noisy-return cluster represented consolidation periods.
\end{itemize}

These clusters overlapped closely with supervised classifier outputs and PCA projections, reinforcing the robustness and reproducibility of regime boundaries. This demonstrates that regime structure is data-driven rather than arbitrarily imposed.

\section{Anomaly Detection}
Although not the central focus of this dissertation, anomaly detection provides valuable insight into price spikes, volume surges and structural breaks. Two approaches were examined.

\subsection{Z-Score Based Outlier Detection}
A simple z-score applied to daily returns flagged extreme movements exceeding $\pm 3$ standard deviations. These outliers often coincided with:
\begin{itemize}
    \item earnings announcements,
    \item macroeconomic news releases,
    \item sudden sentiment shifts.
\end{itemize}

Such points frequently align with regime-transition boundaries and should be treated cautiously by forecasting models.

\subsection{Isolation Forest}
Isolation Forest, which isolates anomalous observations via recursive random partitioning, identified clusters of abnormal activity corresponding to:
\begin{itemize}
    \item volatility explosions,
    \item liquidity drops,
    \item abrupt price gaps.
\end{itemize}

Although optional within the main pipeline, these results highlight structural breaks and periods where predictive models—especially deep learning architectures—face increased risk of failure.

\section{Summary}
Unsupervised learning uncovers meaningful structure in financial markets without requiring labelled data. Key findings include:
\begin{itemize}
    \item K-Means reliably identifies regimes that align with supervised classifications;
    \item PCA reveals latent trend and volatility factors that explain most market variance;
    \item volatility--return clustering exposes clear bull, bear and consolidation regimes;
    \item anomaly detection highlights structural breaks that often precede model underperformance.
\end{itemize}

These insights provide a foundation for comparing supervised, unsupervised and deep learning models under different market states, as explored in subsequent chapters.
