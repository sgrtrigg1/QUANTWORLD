\begin{figure}[H]
    \centering
    \includegraphics[width=\textwidth]{figures/results/AAPL_2025_ols_global.png}
    \caption{Global OLS regression fitted to 2025 AAPL daily closing prices. 
    The single linear trend fails to capture direction changes and exhibits persistent bias, 
    illustrating why global linear models underperform in non-stationary markets.}
    \label{fig:ols-global}
\end{figure}
\begin{figure}[H]
    \centering
    \includegraphics[width=\textwidth]{figures/results/AAPL_2025_ols_first_vs_second.png}
    \caption{Illustration of underfitting: an OLS model trained on the first half of 2025 
    performs poorly when applied to the second half. The divergence highlights structural changes 
    and violates the assumption of stable linear relationships.}
    \label{fig:underfitting}
\end{figure}
\begin{figure}[H]
    \centering
    \includegraphics[width=\textwidth]{figures/results/AAPL_2025_ols_segmented_4.png}
    \caption{Segmented OLS fit across four manually defined market segments. 
    Each segment captures local trend structure far more accurately than the global model, 
    demonstrating the value of piecewise linear modelling.}
    \label{fig:ols-segmented}
\end{figure}
\begin{figure}[H]
    \centering
    \includegraphics[width=\textwidth]{figures/results/AAPL_2025_ols_rolling_fit_win60.png}
    \caption{Rolling OLS regression with a 60-day window. 
    The local regression adapts to changing trends and provides a closer fit to observed prices than the global model.}
    \label{fig:ols-rolling-fit}
\end{figure}
\begin{figure}[H]
    \centering
    \includegraphics[width=\textwidth]{figures/results/AAPL_2025_ols_slopes_win60.png}
    \caption{Sliding-window OLS slope values across 2025. 
    Positive slopes correspond to bullish phases, while negative slopes coincide with drawdown periods.
    This aligns with market regimes and shows how rolling slopes can function as a primitive regime indicator.}
    \label{fig:ols-rolling-slopes}
\end{figure}
\begin{figure}[H]
    \centering
    \includegraphics[width=\textwidth]{figures/results/AAPL_2025_ann_forecast.png}
    \caption{Artificial Neural Network forecast for AAPL (2025). 
    The ANN captures broad upward trends but smooths out volatility, reflecting moderate flexibility 
    but sensitivity to noise.}
    \label{fig:ann-forecast}
\end{figure}
\begin{figure}[H]
    \centering
    \includegraphics[width=\textwidth]{figures/results/AAPL_2025_lstm_forecast.png}
    \caption{LSTM forecast for AAPL (2025). 
    The LSTM adapts more effectively to local temporal structure compared to the ANN, 
    though it still underreacts to sharp fluctuations.}
    \label{fig:lstm-forecast}
\end{figure}
\begin{figure}[H]
    \centering
    \includegraphics[width=\textwidth]{figures/results/AAPL_2025_arima_forecast.png}
    \caption{ARIMA(1,1,1) forecast for AAPL (2025). 
    The ARIMA model fails to capture market reversals or non-linear patterns, resulting in a flat forecast 
    that severely underperforms.}
    \label{fig:arima-forecast}
\end{figure}
\begin{figure}[H]
    \centering
    \includegraphics[width=\textwidth]{figures/results/AAPL_2025_bandit_cum_rewards.png}
    \caption{Cumulative reward for a simple multi-armed bandit trading agent. 
    The declining reward indicates that naive RL approaches struggle in noisy, 
    non-stationary financial environments without proper state representation or risk-aware design.}
    \label{fig:bandit-rewards}
\end{figure}
\begin{figure}[H]
    \centering
    \includegraphics[width=\textwidth]{figures/results/AAPL_2025_anomalies_iforest.png}
    \caption{Isolation Forest anomaly detection on daily returns for AAPL (2025). 
    The model successfully identifies extreme positive and negative return spikes as outliers, 
    corresponding to short-lived volatility shocks.}
    \label{fig:iforest-anomalies}
\end{figure}
\begin{figure}[H]
    \centering
    \includegraphics[width=0.75\textwidth]{figures/results/AAPL_2025_pca_pc1_pc2.png}
    \caption{PCA projection of the feature matrix onto the first two principal components. 
    The distribution highlights patterns in the transformed space, with PC1 capturing most of the variance.}
    \label{fig:pca-projection}
\end{figure}
\begin{figure}[H]
    \centering
    \includegraphics[width=0.75\textwidth]{figures/results/AAPL_2025_pca_scree.png}
    \caption{PCA scree plot showing the explained-variance ratio for the first three components. 
    PC1 accounts for over 80\% of the total variance, justifying dimensionality reduction.}
    \label{fig:pca-scree}
\end{figure}
\begin{figure}[H]
    \centering
    \includegraphics[width=0.75\textwidth]{figures/results/AAPL_2025_kmeans_clusters.png}
    \caption{K-Means clustering of rolling returns and rolling volatility (k=3). 
    The clusters correspond to three intuitive market regimes: 
    (1) low-volatility stable movement, 
    (2) medium-volatility upward drift, 
    (3) high-volatility turbulent periods.}
    \label{fig:kmeans-clusters}
\end{figure}
\begin{figure}[H]
    \centering
    \includegraphics[width=0.65\textwidth]{figures/results/AAPL_2025_confusion_matrix.png}
    \caption{Confusion matrix for the market regime classifier (bull, sideways, bear). 
    The model achieves perfect precision for the sideways regime but struggles to identify 
    bear conditions due to limited training examples.}
    \label{fig:confusion-matrix}
\end{figure}
